\abstract{
Stochastic optimization methods refer to the use and generation of random variables to compute solutions for complex combinatorial optimization problems. In computer science, there remains a major unsolved hypothesis called the \emph{"P versus NP problem"}, which discloses whether every problem whose solutions can be quickly verified, can also be quickly solved for. $P$ stands for polynomial time, which roughly means a set of relatively easy problems to be solved for. $NP$ stands for non-deterministic polynomial time which is a set of difficult problems to be solved for. So, if $P = NP$, this would imply that the difficult set of problems have relatively easily computed solutions. So we apply this onto a \emph{NP-hard} bioinformatics problem called the \emph{Phylogenetic Tree reconstruction}, where we wish to explain the relations of the SARS-CoV-2 Deltacoronavirus epidemic directly from 10 different DNA genomes \emph{(courtesy of the GenBank database owned by National Center for Biotechnology Information (NCBI) of USA, www.ncbi.nlm.nih.gov)}. The issue that arises from having 10 genome samples is that, there are 34,459,425 binary rooted trees to choose from. Therefore we use a metaheuristic called the \emph{Genetic Algorithm}, which specializes in navigating massive search spaces to find the most optimal solution. This genetic algorithm is combined with another popular bioinformatics tool called the \emph{Multiple Sequence Alignment} which is morphed into a method called \emph{SAGA} or Sequence Alignment by Genetic Algorithm. By using this method we are able to compute a cluster-based distance matrix and feed it to the \emph{Neighbor-joining} algorithm, where we can find and produce an optimized tree to solve a NP-hard problem in O(log n) time complexities which are extremely efficient and computationally quick for a problem that would take a lifetime if brute-forced. By following these procedure we were able to form our phylogenetic tree. This research is in our effort to understand the SARS-CoV-2 Deltacoronavirus and see the evolutionary tracks and similarities between the genetic sequences of different species and identify any correlations or outgroups between sequences.
}
