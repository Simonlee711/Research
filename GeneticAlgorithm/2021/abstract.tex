\abstract{
Stochastic optimization methods refer to the use and generation of random variables. This is prevelant in the class of evolutionary algorithms, in which the focus of this paper is specifically that of the \emph{genetic algorithm}, which is used in biology, physics, computer science, and engineering to compute solutions for complex combinatorial optimization problems which we have no other way to calculate for solutions. In computer science, there remains a major unsolved hypothesis called the \emph{"P versus NP problem"}, which discloses whether every problem whose solutions can be quickly solved for, can also be verified quickly. 
\emph{P} stands for \emph{polynomial time}, which roughly means a set of relatively easy problems to be solved for.
\emph{NP} stands for \emph{non-deterministic polynomial time} which is a set of difficult problems to be solved for. 
So, if \emph{P = NP}, this would imply that the difficult set of problems have relatively easily computed solutions. This theoretical question is the foundation of the \emph{genetic algorithm}, which aims to find the best solution, by the evolutionary process of \emph{natural selection}, at exponentially fast time complexities which would take years to compute if brute-forced.  
The essence of this algorithm is applied to research a condensed population, in order to design a distribution strategy of vaccines to minimize the number of infenctions within an epidemic outbreak.
}
